\documentclass[a4paper, DIV=17, twocolumn]{scrartcl}
\usepackage{tikz}
\usepackage{environ}
\usepackage{adjustbox}
\makeatletter
\newsavebox{\measure@tikzpicture}
\NewEnviron{scaletikzpicturetowidth}[1]{%
  \def\tikz@width{#1}%
  \def\tikzscale{1}\begin{lrbox}{\measure@tikzpicture}%
  \BODY
  \end{lrbox}%
  \pgfmathparse{#1/\wd\measure@tikzpicture}%
  \edef\tikzscale{\pgfmathresult}%
  \BODY
}
\makeatother

\title{Atlas of tropical elliptic curves}
\author{Laura Casabella\\Lars Kastner\\Raluca Vlad}

\begin{document}
\maketitle
\tableofcontents
\section{Documentation}
\subsection{Point configuration}
The point configuration we are working with is given as the rows of the following matrix:
\[
   \begin{array}{c|ccccc}
      \textbf{A} & 1 & 0 & 0 & 0 & 0\\
      \textbf{a} & 1 & 0 & 0 & 0 & 1\\
      \textbf{B} & 1 & 0 & 0 & 1 & 0\\
      \textbf{b} & 1 & 0 & 0 & 1 & 1\\
      \textbf{C} & 1 & 0 & 0 & 2 & 0\\
      \textbf{c} & 1 & 0 & 0 & 2 & 1\\
      \textbf{D} & 1 & 0 & 1 & 0 & 0\\
      \textbf{d} & 1 & 0 & 1 & 0 & 1\\
      \textbf{E} & 1 & 0 & 1 & 1 & 0\\
      \textbf{e} & 1 & 0 & 1 & 1 & 1\\
      \textbf{F} & 1 & 0 & 2 & 0 & 0\\
      \textbf{f} & 1 & 0 & 2 & 0 & 1\\
      \textbf{G} & 1 & 1 & 0 & 0 & 0\\
      \textbf{g} & 1 & 1 & 0 & 0 & 1\\
      \textbf{H} & 1 & 1 & 0 & 1 & 0\\
      \textbf{h} & 1 & 1 & 0 & 1 & 1\\
      \textbf{I} & 1 & 1 & 1 & 0 & 0\\
      \textbf{i} & 1 & 1 & 1 & 0 & 1\\
      \textbf{J} & 1 & 2 & 0 & 0 & 0\\
      \textbf{j} & 1 & 2 & 0 & 0 & 1
   \end{array}
\]
The rows are labelled in the first column with $A-T$. We will use these
labels later to present the simplices of triangulations in a compact format. For
example the string
\[
GJKLR\ =\ [6,9,10,11,17]
\]
encodes the simplex that is the convex hull of the corresponding rows in the
matrix.

The rows of this matrix are the lattice points of the Cayley polytope
$C(2\Delta_3, 2\Delta_3)$, except it was necessary to project this down to
arrive at something full-dimensional for \texttt{mptopcom} to work.

\subsection{ID}
The ID's of the graphs are just a continuous numbering of these for easier
reference.

\subsection{Canonical Hash (CH)}
In \texttt{polymake} one can compute the \texttt{canonical\_hash} of a graph,
which in the background is computed by either \texttt{nauty} or \texttt{bliss}.
Essentially two graphs with different hash are never isomorphic, and except for
pathological cases two graphs with the same hash will be isomorphic. In the
case of the graphs in this document the canonical hash is unique.

\subsection{Triangulation}
For every graph we also recorded a triangulation giving rise to this graph.

\section{Graph classification}
\setlength{\parindent}{0pt}
\input{inner.tex}

\end{document}
